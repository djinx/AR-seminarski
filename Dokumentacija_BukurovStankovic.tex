\documentclass[a4paper, 12pt]{article}

% srpski jezik, utf8
\usepackage[serbian]{babel}
\usepackage[utf8]{inputenc}

% podrska za graficke elemente
\usepackage{graphicx}

% podrska za margine i sl.
\usepackage{geometry}

% podrska za boje
\usepackage{xcolor}

% podrska za stavljanje hyperlink-ova u sadrzaj
\usepackage{hyperref} 

% za [H] u figure
\usepackage{float} 

% za vertikalno spajanje celija u tabeli
\usepackage{multirow}


% meta-informacije o tekstu
\author{Bukurov Anja 1082/2016\\Vojislav Stanković 1080/2016}
\title{Podaci o muzejima}
\date{\today}

\begin{document}

\begin{titlepage}
	\centering
	\noindent{\Large Matematički fakultet\\Univerzitet u Beogradu}
	
	\vspace{0.3\textheight}
	
	{\Huge DP procedura}
	\\~
	\\
	{\Large — dokumentacija —}
	
	\vfill
	
	{
		\Large 
		\begin{tabular}{|l|l|l|}
			\hline
			\multirow{2}{*}{Studenti} & Bukurov Anja        & 1082/2016\\ \cline{2-3}
			                          & Stanković Vojislav 	& 1080/2016\\
			\hline
			Predmet                   & \multicolumn{2}{l|}{Automatsko rezonovanje}\\
			\hline
			Školska godina         & \multicolumn{2}{l|}{2016/2017}\\
			\hline
			Nastavnik                 & \multicolumn{2}{l|}{dr Filip Marić}\\
			\hline
			Datum                     & \multicolumn{2}{l|}{\today}\\
			\hline
		\end{tabular}
	}
\end{titlepage}
\newpage

\section{Opis rada}

U ovom radu opsiano je kako smo implementirali DP proceduru u programskom jeziku C++. Opisali smo funckije korišćene za samu porceduru, kao i pomoćne funkcije za rukovanje literalima i klauzama.

Sama procedura implementirana je tako da redom eliminiše jednu po jednu promenljivu iz formule i ako uspe da ih sve eliminiše program vraća SAT, a u suprotom UNSAT.


\section{Literali, promenljive i klauze}

\noindent Literali i promenljive su predstavljeni tipom \verb|unsigned|:

\begin{verbatim}
  typedef unsigned Var;
  typedef unsigned Literal;
\end{verbatim}


\noindent Klauza je predstavljena skupom literala: 
\begin{verbatim}
  typedef set<Literal> Clause;
\end{verbatim}

\noindent Formula je skup klauza:
\begin{verbatim}
  typedef set<Clause> FormulaCNF;
\end{verbatim}

\noindent Napravili smo nabrojivi tip \verb|Polarity| koji služi za određivanje polariteta literala: 
\begin{verbatim}
  enum Polarity {POSITIVE, NEGATIVE};
\end{verbatim}


\subsection{Funkcije}

\noindent Od funkcija za rukovanje literalima i promenljivama, implementirali smo sledeće:

\begin{description}
	\item[Literal litFromVar(Var v, Polarity p);] ~ 
	\begin{itemize}
		\item funkcija za konverziju promenljive u literal datog polariteta
		\item implementirano pomoću bitovskih operatora
		\item pozitivni literali predstavljaju se parnim brojevima, a negativni neparnim
		\item pozitivan literal dobija se šiftovanjem vrednosti promenljive \verb|v| u levo
		\item negativan literal dobija se šiftovanjem vrednosti promenljive \verb|v| u levo i primeniom bitovske disjunkcije na dobijenu vrednost i 1
	\end{itemize}

	\item[Var varFromLit(Literal l);] ~ 
	\begin{itemize}
		\item funkcija za konverziju literala proizvoljnog polariteta u promenljivu
		\item implementirano pomoću bitovskih operatora 
		\item \verb|l| se šiftuje u desno čime se odbacuje najniži bit koji je označavao polaritet literala
	\end{itemize}
	
	\item[bool isPositive(Literal l);] ~ 
	\begin{itemize}
		\item funkcija kojom se ispituje da li je literal pozitivan 
		\item literal je pozitivan ako je broj paran, tj. ako je najniži bit 0
		\item provera se vrši primenom bitovske konjunkcije na \verb|l| i 1 i negacijom dobijene vrednosti
	\end{itemize}

	\item[bool isNegative(Literal l);] ~ 
	\begin{itemize}
		\item funkcija kojom se ispituje da li je literal negativan 
		\item literal je negativan ako je broj neparan, tj. ako je najniži bit 1
		\item provera se vrši primenom bitovske konjunkcije na \verb|l| i 1
	\end{itemize}

	\item[Literal oppositeLiteral(Literal l);] ~ 
	\begin{itemize}
		\item funkcija za dobijanje suprotnog literala
		\item suprotan literal dobija se invertovanjem najnižeg bita tj. primenom bitovske eksluzivne disjunkcije na \verb|l| i 1
	\end{itemize} 
	
	\item[int intFromLit(Literal l);] ~ 
	\begin{itemize}
		\item funkcija za dobijanje celobrojne vrednosti od literala
		\item proverava se polaritet literala \verb|l| 
		\item ukoliko je pozitivan, na vrednost dobijenu pozivom funkcije \verb|varFromLit(l)| dodaje se 1 i sve se kastuje u \verb|int|
		\item ukoliko je negativan, na vrednost dobijenu pozivom funkcije \verb|varFromLit(l)| dodaje se 1 i sve se kastuje u \verb|int| a potom negira
	\end{itemize}

	\item[Literal litFromInt(int i);] ~ 
	\begin{itemize}
		\item funkcija za dobijanje literala od celobrojne vrednosti
		\item ukoliko je broj \verb|i| pozitivan, pravi se pozitivan literal pozivom funkcije \verb|litFromVar(i - 1, POSITIVE)|
		\item ukoliko je broj \verb|i| negativan, pravi se negativan literal pozivom funkcije \verb|-litFromVar(i - 1, NEGATIVE)|
	\end{itemize}
\end{description}






\section{Klasa DPSolve}

Klasa DPSolve služi za primenu DP procedure na određenu formulu. Klasa sadrži dva privatna polja:   \verb|FormulaCNF _formula;| koje predstavlja formulu čiju zadovoljivost proveravamo i \verb|unsigned _num;| koje predstavlja broj promenljivih u formuli.
\\
\\
\noindent Od javnih metoda, klasa sadrži:

\begin{description}
	\item[DPSolve(const FormulaCNF \& f, unsigned num);] ~ 
	\begin{itemize}
		\item konstruktor
	\end{itemize}
	
	\item[bool contains(const Clause \& c, Literal l);] ~
	\begin{itemize}
		\item funkcija koja proverava da li klauza c sadrzi literal l
		\item implementarano pomoću metode za pretragu skupa find()
	\end{itemize}

	\item[bool resolution(Var v, const Clause \& c1, const Clause \& c2, Clause \& r);]
	~ 
	\begin{itemize}
		\item funkcija koja primenjuje pravilo rezolucije nad klauzama \verb|c1| i \verb|c2| po literalu \verb|l| 
		\item klauza \verb|c1| sadrži pozitivan a \verb|c2| negativan literal \verb|l|
		\item parametar \verb|v| služi za lakše određivanje pozitivnog i negativnog literala \verb|l|
		\item \verb|r| je rezultujuća klauza koja ne sadrži \verb|l| 
		\item na početku, klauza \verb|r| je jednaka klauzi \verb|c1| a onda se iz nje briše pozitivan literal \verb|l|
		\item zatim se za svaki literal \verb|k| iz klauze \verb|c2| proverava da li je jednak negativnom literalu \verb|l| - njih preskačemo a za ostale proveravamo da li klauza \verb|r| sadrži literal suprotan literalu \verb|k|
		\item ako je to ispunjeno onda je klauza \verb|r| tautologija i vraća se \verb|false| jer ne može dalje da se primenjuje rezolucija
		\item ako u \verb|r| nema suprotnog literala literalu \verb|k| onda se on dodaje u klazu \verb|r|
	\end{itemize}

	\item[bool eliminate(Var v);] ~
	\begin{itemize}
		\item funkcija koja uklanja iskazno slovo \verb|v| iz formule primenom pravila rezolucije
		\item traže se klauze formule \verb|_formula| koje sadrže pozitivan literal koji odgovara promenljivoj \verb|v| i ne sadrže negativan literal koji odgovara promenljivoj \verb|v| i takve klauze dodaje u novu formulu  \verb|nf|
		\item za svaku pronađenu klauzu  \verb|c1| koja sadrži pozitivan literal traži se klauza  \verb|c2| koja sadrži negativan literal koji odgovara promenljivoj  \verb|v| i nad njima se poziva funkcija  \verb|resolve(v, c1, c2, r)|
		\item ukoliko neki poziv funkcije  \verb|resolve(v, c1, c2, r)| vrati  praznu klauzu \verb|r| funkcija  \verb|eliminate(v)| vratiće \verb|false| čime se označava da nema više šta da se eliminiše, u suprotnom se klauza  \verb|r| dodaje u novu formulu
		\item ukoliko nije dobijena nijedna prazna klauza,  \verb|eliminate(v)| vraća \verb|true| a privatno polje \verb|_formula| se menja u novodobijenu formulu
	\end{itemize}

	\item[bool checkIfSat();] ~
	\begin{itemize}
		\item funkcija koja proverava zadovoljivost formule
		\item za sve promenljive, redom, proverava se da li mogu da se eliminišu \verb|eliminate(v)|
		\item ukoliko  \verb|eliminate(v)| vrati  \verb|false| to znači da je izvedena prazna klauza i formula je nezadovoljiva, a funkcija vraća \verb|false|
		\item ukoliko su sve promenljive uspešno eliminisane, vraća se \verb|true|
	\end{itemize}

	\item[int skipSpaces(istream \& istr);] ~
	\begin{itemize}
		\item funkcija koja preskače beline pri učitavanju fajla u DIMACS formatu
	\end{itemize}

	\item[int skipRestOfLine(istream \& istr);] ~
	\begin{itemize}
		\item funkcija koja preskače ostatak reda pri učitavanju fajla u DIMACS formatu
	\end{itemize}
	
	\item[bool inDimacs(FormulaCNF \& f, unsigned \& num\_of\_vars, istream \& istr);] ~ 
	\begin{itemize}
		\item funkcija koja učitava fajl u DIMACS formatu
		\item prilikom čitanja se postavljaju vrednosti privatnih polja \verb|_formula| i \verb|_num|
	\end{itemize}
\end{description}


\section{Main funkcija}

U okviru main funckije poziva se funkcija za čitanje fajla u DIMACS formatu \verb|inDimacs(f, num, cin)|. Zatim se poziva konstruktor klase \verb|DPSolver(f, num)|. Na kraju se poziva metoda klase \verb|checkIfSat()| koja će vrati  \verb|true| ako je formula zadovoljiva, a u suprotnom vraća  \verb|false|. Na osnovu povratne vrednosti funkcije, na standardni izlaz se ispisuje \verb|SAT| odnosno \verb|UNSAT|.



\section{Test primeri}

U tabeli \ref{tab:testPrimeri} prikazali smo koliko memorije program koristi za ulaze sa različitim brojem promenljivih i klauza.


\begin{table}[H]
	\centering
	\label{tab:testPrimeri}
	\caption{Primeri rada programa}
	\begin{tabular}{|c|c|c|c|} 
		\hline
		promenljivih & klauza & memorija & SAT/UNSAT \\
		\hline
		729 & 11788 & 3.44 GB (Memory out) & SAT \\
		\hline
		16 & 60 & 378 KB & SAT \\
		\hline
		22 & 70 & 1.84 GB & SAT \\
		\hline
		75 & 325 & Memory out & SAT \\
		\hline
		32 & 97 & 2.31 MB & SAT \\
		\hline
		20 & 44 & 223 KB & SAT \\
		\hline
		30 & 95 & 3.44 GB (Memory out) & UNSAT \\
		\hline
		10  & 28 & 133 KB & UNSAT \\
		\hline
		6 & 11 & 37 KB & UNSAT \\
		\hline 
		24 & 49 & 338 KB & UNSAT \\
		\hline
		50 & 218 & Memory out & UNSAT \\
		\hline
		100 & 430 & Memory out & UNSAT \\
		\hline
	\end{tabular}
\end{table}

\end{document}